\section{Methods}
For temporal networks, we model exposure dynamically. 
We choose a simple setting in which the probability of exposure is \textit{i.i.d.} in time, but varies 
between node pairs. For simplicity, all derivations are done undirected networks. 
However, this can easily be extended to directed networks if need be. 
Additionally, in previous work the exposure prior was quite weak as the propensity to be exposed for node $i$ is the same for each node $j$
it could be exposed to. Here, we increase the expressiveness of $\mu$ by making it a $\tilde{K}$-dimensional vector. 
As is the case with $u$ and $v$, $\mu$ can be interpreted as a mixed membership community vector, $\textit{i.e.}$
we are now able to model exposure communities as well as communities due to affinity.\\

\paragraph{Resulting model}
These two changes result in the following prior: 
\begin{equation}\label{eqn:expMarkov}
    \begin{aligned}
        p(\textbf{Z}) &= \prod_t \prod_{i<j} p(\Zijt) \\
                      &= \prod_t \prod_{i<j} \muij^{\Zijt} (1-\muij)^{1-\Zijt}\quad.
    \end{aligned}      
\end{equation}

We can see that the Bernulli prior on \textbf{Z} now depends on $\mu_i^T \mu_j$, that is the similarity between the exposure communities
of $i$ and $j$. 
Analogously to the static model, we choose the likelihood as:
\begin{equation}
    \begin{aligned}
        p(\textbf{A} \mid \theta, \textbf{Z}) = \prod_t \prod_{i<j} Pois(\Aijt ; \lambdaij)^{\Zijt} \delta(\Aijt)^{(1-\Zijt)}\quad.
    \end{aligned}      
\end{equation}
