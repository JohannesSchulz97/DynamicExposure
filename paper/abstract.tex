\begin{abstract}
    When modeling interactions between individuals in a network, probabilistic generative models often assume that 
    any interaction could in principle exist. If an interaction is not observed, 
    this is attributed to the fact that the chosen mechanism for tie formation is not occuring. 
    For instance, two nodes may not have a high affinity. 
    However, these models ignore the possibility that nodes may not have been exposed to each other.
    That prevents them from interacting in the first place, regardless the mechanism of tie formation.
    In particular, such approaches fail to capture situations where individuals may indeed be compatible and thus interact, 
    if they were exposed to each other.
    
    While this problem is known in recommender systems, it has only recently been investigated in network models. 
    This study has been limited to static networks, where interactions do not evolve in time. 
    As many real-world networks are constantly changing, interactions and exposure between nodes need to be modeled 
    within the framework of temporal networks. Here, we introduce two approaches that tackle this problem by extending
    the concept of exposure to a dynamical setting. They both rely on a Bayesian probabilistic approach that considers 
    community structure as the main underlying mechanism for tie formation. 
  
    By properly modeling how exposure between nodes changes in time, we obtain strong community reconstruction results
    for both synthetic and real data of recorded proximity interactions between individuals. 
    On the latter, we additionally observe a significant improvement in terms of reconstructing unobserved interactions 
    compared to a standard model that does not take exposure into account.
\end{abstract}