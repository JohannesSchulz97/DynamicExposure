%%%%%%%%%%%%%%%%%%%%%%%%%%%%%%%%%%%%%%%%%%%%%%%%%%%%%%%%%%%%%%%%%%%%%%%%
%    INSTITUTE OF PHYSICS PUBLISHING                                   %
%                                                                      %
%   `Preparing an article for publication in an Institute of Physics   %
%    Publishing journal using LaTeX'                                   %
%                                                                      %
%    LaTeX source code `ioplau2e.tex' used to generate `author         %
%    guidelines', the documentation explaining and demonstrating use   %
%    of the Institute of Physics Publishing LaTeX preprint files       %
%    `iopart.cls, iopart12.clo and iopart10.clo'.                      %
%                                                                      %
%    `ioplau2e.tex' itself uses LaTeX with `iopart.cls'                %
%                                                                      %
%%%%%%%%%%%%%%%%%%%%%%%%%%%%%%%%%%
%
%
% First we have a character check
%
% ! exclamation mark    " double quote  
% # hash                ` opening quote (grave)
% & ampersand           ' closing quote (acute)
% $ dollar              % percent       
% ( open parenthesis    ) close paren.  
% - hyphen              = equals sign
% | vertical bar        ~ tilde         
% @ at sign             _ underscore
% { open curly brace    } close curly   
% [ open square         ] close square bracket
% + plus sign           ; semi-colon    
% * asterisk            : colon
% < open angle bracket  > close angle   
% , comma               . full stop
% ? question mark       / forward slash 
% \ backslash           ^ circumflex
%
% ABCDEFGHIJKLMNOPQRSTUVWXYZ 
% abcdefghijklmnopqrstuvwxyz 
% 1234567890
%
%%%%%%%%%%%%%%%%%%%%%%%%%%%%%%%%%%%%%%%%%%%%%%%%%%%%%%%%%%%%%%%%%%%
%
\documentclass[12pt]{iopart}
\newcommand{\gguide}{{\it Preparing graphics for IOP Publishing journals}}
%Uncomment next line if AMS fonts required
%\usepackage{iopams}  
\begin{document}

\title{Investigating the impact of exposure in probabilistic models for dynamical networks}

\author{Johannes Schulz$^1$, Caterina De Bacco$^1$, Marco Baity-Jesi$^2$}


\address{$^1$ Max Planck Institute for Intelligent Systems, Cyber Valley, Tuebingen, 72076, Germany}
\address{$^2$ Eawag, \"Uberlandstrasse 133, CH-8600 D\"ubendorf, Switzerland}
\ead{caterina.debacco@tuebingen.mpg.de, marco.baityjesi@eawag.ch}

%\address{IOP Publishing, Temple Circus, Temple Way, Bristol BS1 6HG, UK}
%\ead{submissions@iop.org}
\vspace{10pt}

\date{\today}
%\begin{indented}
%\item[]September 2022
%\end{indented}

\begin{abstract}
  When modeling interactions between individuals in a network, probabilistic generative models often assume that 
  any interaction could in principle exist. If an interaction is not observed, 
  this is attributed to the fact that the chosen mechanism for tie formation is not occuring. 
  For instance, two nodes may not have a high affinity. 
  However, these models ignore the possibility that nodes may not have been exposed to each other.
  That prevents them from interacting in the first place, regardless the mechanism of tie formation.
  In particular, such approaches fail to capture situations where individuals may indeed be compatible and thus interact, 
  if they were exposed to each other.
  
  While this problem is known in recommender systems, it has only recently been investigated in network models. 
  This study has been limited to static networks, where interactions do not evolve in time. 
  As many real-world networks are constantly changing, interactions and exposure between nodes need to be modeled 
  within the framework of temporal networks. Here, we introduce two approaches that tackle this problem by extending
  the concept of exposure to a dynamical setting. They both rely on a Bayesian probabilistic approach that considers 
  community structure as the main underlying mechanism for tie formation. 

  By properly modeling how exposure between nodes changes in time, we obtain strong community reconstruction results
  for both synthetic and real data of recorded proximity interactions between individuals. 
  On the latter, we additionally observe a significant improvement in terms of reconstructing unobserved interactions 
  compared to a standard model that does not take exposure into account.
\end{abstract}

%
% Uncomment for keywords
%\vspace{2pc}
%\noindent{\it Keywords}: XXXXXX, YYYYYYYY, ZZZZZZZZZ
%
% Uncomment for Submitted to journal title message
%\submitto{\JPA}
%
% Uncomment if a separate title page is required
%\maketitle
% 
% For two-column output uncomment the next line and choose [10pt] rather than [12pt] in the \documentclass declaration
%\ioptwocol
%

\maketitle
%\newpage
 
\section{Template Summary}

\begin{itemize}
  \item there is an incompatibility between amsmath.sty and iopart.cls
  which cannot be completely worked around. If your article relies on commands in
  amsmath.sty that are not available in iopart.cls, you may wish to consider using a
  different class file.
  \item The ‘master LATEX file
  must read in all other LATEX and figure files from the current directory
  \item The words table and figure should be written in full and not abbreviaged to tab.
  and fig. Do not include ‘eq.’, ‘equation’ etc before an equation number or ‘ref.’
  ‘reference’ etc before a reference number.
  \item All journals to which this document applies allow
  the use of either the Harvard or Vancouver system
\end{itemize}

\section{Methods}



%\bibliographystyle{abbrv}
%\bibliography{references}

\end{document}

