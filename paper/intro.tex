\section{Introduction}

Networks are a popular and effective representation of large-scale complex systems, 
encompassing socio-economical relations, the human brain, cell metabolism,
ecosystems, informational infrastructure, and many more \cite{newman2018networks,newman2003structure}. 
A network or graph consists of a set of nodes representing distinct entities 
as well as a set of edges which represent interactions between these entities.
This basic model is often extended to account for the specific properties of the system at hand.
For cases in which not only interactions, but also their intensity is observed, weighted networks are used. 
Networks can be undirected, if interactions are symmetric, or directed if that is not the case.
Additionally, many real-world systems are not static but evolve over time.
This can be modeled through temporal networks \cite{masuda2016guide,holme2012temporal}.

To understand the hidden patterns and processes in complex systems, 
it is helpful to further abstract a complex network structure composed of many microscopic variables into a lower-dimensional representation that coarsens the system. 
Probabilistic generative models \cite{goldenberg2010survey} are mathematical approaches that aim at doing so by investigating the generative process, 
\textit{i.e.} how exactly interactions between nodes are taking place, using a set of simplifying assumptions.
Here, the challenge is to come up with a model that captures reality as accurately as possible 
while remaing relatively simple. 
One quite intuitive approach is to suppose that nodes can be grouped into equivalence classes and that interactions between them
only depend on their class memberships. This assumption is called \textit{Stochastic Equivalence} and lies at the heart of the
\textit{Stochastic Block Model} (\sbm) \cite{holland1983stochastic}, a popular and flexible probabilistic model for network data. 

As the \sbm\ assumes that nodes can be clustered into groups, it naturally lends itself to the task of community detection. 
Community detection is an inference task that aims at clustering similar nodes into communities 
\cite{fortunato2010community, javed2018community, lancichinetti2009community}. 
Two nodes are considered similar if they interact in a similar way.
In \Cref{assortative_sbm}, we show a synthetic network generated with a \sbm\ as well as the corresponding adjacency matrix. 
An adjacency matrix represents a graph as a matrix $A$, such that the entry $\Aij$ models the weight of an edge between nodes $i$ and $j$.
In the example in \Cref{assortative_sbm}, we can observe an assortative structure \cite{fortunato2016community}, 
\textit{i.e.} nodes that belong to the same community 
have a higher likelihood of interacting than nodes that belong to different communities. 

One main property observed in real networks is sparsity, \textit{i.e.} the number of existing interactions is linear in the number of nodes. 
This means that one observes only a small fraction among the many possible interactions between pairs of nodes.
For instance, online social networks can contain billions of users, but each user only interacts
with a tiny fraction of that. The observed non-existing interactions are traditionally attributed to a low affinity between nodes
(Alice not interacting with Jack implies that Alice does not like Jack).
However, often the actual reason is not a low affinity, but rather the fact that the two users have never met, 
\textit{i.e.} never been \textit{exposed} to each other. \Cref{exposure_mechanism} visualizes this exposure mechanism. 
The idea of exposure has been explored in the context of recommender systems \cite{liang2016modeling,chuklin2015click,wang2016learning,yang2018unbiased}. 
However, adapting these techniques to social networks is non-trivial and the investigation of this problem is still missing. 
Nevertheless, it is crucial to do so, as non-existing links contain important information which can only partially be captured 

by existing approaches \cite{ball2011efficient,de2017community,schein2016bayesian,zhao2012consistency}.
In previous work \cite{static_exp}, we explored the idea of including exposure as an additional binary latent variable to the \sbm.
This was done for static networks, \textit{i.e.} settings where we only observe one network sample and assume that the network does not change over time. 
In practice though, many real networks are constantly evolving. 
This can have dramatic effects in how the exposure mechanism acts in these systems. 
If exposure models physical proximity and we consider social networks, we can for instance observe that exposure is not static at all. 
Also, once people are exposed to each other, they are more likely to stay exposed.
Modelling exposure in a static way fails to take these two observations into account.
Hence, it is paramount to investigate approaches that model exposure in dynamical settings and to understand its interplay with an underlying community structure. 
This is the main goal of this project.

To this end we develop two methods, which model exposure in a conceptually very different way. 
Through link prediction experiments, we obtain that one of them, \expm, 
is better able to capture the true generative process of social networks than the same model without exposure \cite{de2017community}.
\expm\ models both exposure as well as affinity through distinct communities.
We show that seperating these two concepts, which are typically implicitly mixed in traditional models, yields promising inference results.

This rest of this thesis is structured as follows: 
In chapter \ref{methods}, we provide the theoretical background regarding \sbm s and the exposure mechanism. 
Then, we introduce two approaches, \expm\ and \exph, that adapt the idea of exposure to a dynamical setting,
as well as the inference algorithm used to infer the optimal latent variables. 
The respective derivations are shown in appendix \ref{deriv}.
In chapter \ref{analysis}, the two methods are evaluated on synthetic and on real data. 
As real data, we use small temporal social networks collected  in different environments 
\cite{primaryschool, highschool, hospital, workplace01, workplace02}
by the \textbf{SocioPatterns} collaboration (\url{http://www.sociopatterns.org}).
We conclude by analysing the obtained results and by investigating potential avenues for future research.
